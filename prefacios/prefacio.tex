\thispagestyle{empty}

\begin{center}
{\large\bfseries Programación eficiente de un algoritmo de procesamiento de la actividad cerebral \\ Implementación y optimización mediante paralelización de un algoritmo de análisis fractal de matrices binarias }\\
\end{center}
\begin{center}
Miguel Ángel Posadas Arráez\\
\end{center}

%\vspace{0.7cm}

\vspace{0.5cm}
\noindent{\textbf{Palabras clave}: \textit{Box counting, análisis fractal de matrices binarias, programación paralela, programación GPU}
\vspace{0.7cm}

\noindent{\textbf{Resumen}\\
El procesamiento de datos es una parte esencial en el estudio de diversas enfermedades neurodegenerativas. No obstante, las técnicas actuales para la adquisión (resonancia magnética, encefalografía etc...) generan grandes volumenes de datos, lo que conlleva que su procesamiento tenga un gran coste computacional. El propósito de este trabajo es la programación eficiente del método conocido como Box counting para el procesamiento de matrices binarias.  Para la toma de tiempos y con el objetivo de poder hacer diversas comparaciones hemos utilizado dos plataformas con software y hardware diferentes. En un principio, el algoritmo venía programado en Matlab, con el objetivo de obtener una versión más eficiente de este algoritmo implementé una versión secuencial en C++ y posteriormente traté de aprovechar el paralelismo dado por los procesadores multi-core introduciendo la tecnología OpenACC obteniendo aceleraciones de hasta 15.7x. Finalmente, para la obtención de mejores aceleraciones, utilizamos el paradigma de computación de propósito general en unidades de procesamiento gráfico (GPGPU) para aprovechar la arquitectura \textit{Single Instruction, Multiple Data} (SIMD) con el que obtenimos aceleraciones de hasta 38.48x .
	

\cleardoublepage

\begin{center}
	{\large\bfseries Same, but in English}\\
\end{center}
\begin{center}
	Miguel Ángel Posadas Arráez\\
\end{center}
\vspace{0.5cm}
\noindent{\textbf{Keywords}: \textit{open source}, \textit{floss}
\vspace{0.7cm}

\noindent{\textbf{Abstract}\\


\cleardoublepage

\thispagestyle{empty}

\noindent\rule[-1ex]{\textwidth}{2pt}\\[4.5ex]

D. \textbf{Tutora/e(s)}, Profesor(a) del ...

\vspace{0.5cm}

\textbf{Informo:}

\vspace{0.5cm}

Que el presente trabajo, titulado \textit{\textbf{Chief}},
ha sido realizado bajo mi supervisión por \textbf{Estudiante}, y autorizo la defensa de dicho trabajo ante el tribunal
que corresponda.

\vspace{0.5cm}

Y para que conste, expiden y firman el presente informe en Granada a Junio de 2018.

\vspace{1cm}

\textbf{El/la director(a)/es: }

\vspace{5cm}

\noindent \textbf{(nombre completo tutor/a/es)}

\chapter*{Agradecimientos}




