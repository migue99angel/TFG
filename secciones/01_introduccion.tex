\chapter{Introducción}


El cálculo de la dimensión fractal de un conjunto de datos es una técnica con una gran multitud de aplicaciones, desde el ámbito económico, -- incluyendo su relación con la premisa del Bitcoin --  {\cite{delfin2016fractal}}, hasta el campo de la medicina y estudio del cuerpo humano, siendo este cálculo útil en la detección de tumores cerebrales {\cite{iftekharuddin2009fractal}}.

La dimensión fractal tiene su cabida en la neurociencia, siendo útil para el procesamiento de la actividad cerebral con el fin de poder estudiar diversas patologías neurodegenerativas \cite{fernandez2001use}. Sin embargo las técnicas actuales para la adquisición de este tipo de datos tales como las electroencefalografías o las resonancias magnéticas, generan grandes volúmenes de datos que implican un alto coste computacional, ya que en función de la dimensión del conjunto de datos, el orden del algoritmo aumenta de manera exponencial, siendo O(n) para conjuntos de datos de una dimensión y O($n^{4}$)  para matrices de cuatro dimensiones.

El propósito de este proyecto es el de la implementación eficiente del algoritmo conocido como Box counting. Para ello, partiendo del algoritmo, inicialmente escrito en el lenguaje de programación Matlab se desarrolla una versión secuencial, escrita en C++ que sirve como base para utilizar diversas tecnologías,como OpenACC y CUDA, con el fin de aprovechar el paralelismo que proporcionan los microprocesadores actuales así como la arquitectura SIMD (\textit{Single Instruction Multiple Data}) presente en las tarjetas gráficas.


\section{Antecedentes}

Si bien es cierto que la implementación eficiente del cálculo de la dimensión fractal ha sido objeto de estudio por varios autores, en ellos se les da un enfoque a otros tipos de conjuntos de datos.\\ En \cite{JIMENEZ20121229} se utiliza una librería incluida en Microsoft Visual Studio para la implementación de una versión multihilo y CUDA para una implementación utilizando la tarjeta gráfica, sin embargo, el algoritmo es aplicado a imágenes(2D) y modelos(3D) en los que los píxeles toman diversos valores, pudiendo ser blanco, negro o alguna tonalidad concreta de gris. En el citado estudio llegan a obtener aceleraciones de hasta 28x con respecto a la versión secuencial. 
\\
En \cite{10.1007/978-3-030-64616-5_8} realizan un enfoque muy parecido al utilizado en este estudio para estudiar un problema un poco distinto, utilizan OpenMP y CUDA para realizar el cálculo de la dimensión fractal sobre curvas definidas analíticamente, llegando a obtener acelereaciones de hasta 58x gracias al uso de la tarjeta gráfica.
\\
Finalmente, \cite{de2020fast} utiliza la misma metodología que se va a utilizar en este trabajo para el estudio de un algoritmo de Box counting diferencial, utilizando OpenACC para el desarrollo de una versión capaz de aprovechar los múltiples núcleos del microprocesador, con la que obtiene aceleraciones de hasta 6x. Y usando CUDA para la implementación de una versión para la GPU, con la que obtiene aceleraciones de hasta 52x con respecto a la versión secuencial.

\section{Objetivos}

Se parte de un algoritmo inicial escrito en Matlab \cite{unknown-author-2008}, y modificado por el tutor del proyecto, el Profesor D. Juan Ruiz de Miras, para que el algoritmo sea capaz de analizar matrices en 4 dimensiones. El objetivo del proyecto es el estudio del algoritmo para detectar las zonas del algoritmo que consumen más tiempo de ejecución, y una vez hayan sido detectados estas zonas, optimizarlas. Para ello se utilizarán tecnologías tales como, OpenACC y CUDA para la parelización del algoritmo.\\

\section{Metodología}

Este Trabajo de Fin de Grado no es un típico trabajo de ingeniería, sino que es un trabajo experimental, por lo que cuenta con una estructura y una metodología de trabajo distinta. En este apartado se trata de definir la metodología a seguir, que será la siguiente: \\

\begin{itemize}
    \item Primero se realizará un estudio teórico del algortimo del que se parte, con el objetivo de comprender correctamente su funcionamiento.
    \item Se implementará una versión secuencial en el lenguaje de programación C++.
    \item Se realizará una etapa de experimentación y toma de tiempos 
    \item Se realizará un estudio del algortimo secuencial con herramientas de profiling para detectar en que zonas del código se consume más tiempo, con el fin de optimizarlas.
    \item Partiendo del código secuencial y del estudio realizado con el profiler, se añadirán las directivas de OpenACC necesarias para tratar de optimizar el algortimo mediante el uso de la CPU multinúcleo.
    \item Se realizará una nueva etapa de experimentación y toma de tiempos
    \item Partiendo del código secuencial y del estudio realizado con el profiler, se utilizará la plataforma de programación CUDA, para la implementación de una versión para la GPU.
    \item Se realizará una última etapa de experimentación y toma de tiempos 
    \item Se realizará un análisis y discusión de los resultados obtenidos
    \item Se redactará una memoria (este documento) documentando todo el proceso seguido, que además, recogerá los resultados obtenidos y las conclusiones a las que se han llegado.
\end{itemize}
 
