\chapter{Metodología}

Este Trabajo de Fin de Grado no es un típico trabajo de ingeniería, sino que es un trabajo experimental, por lo que cuenta con una estructura y una metodología de trabajo distinta. En este apartado se trata de definir la metodología a seguir, que será la siguiente: \\

\begin{itemize}
    \item Primero se realizará un estudio teórico del algoritmo del que se parte, con el objetivo de comprender correctamente su funcionamiento.
    \item Se implementará una versión secuencial en el lenguaje de programación C++.
    \item Se realizará una etapa de experimentación y toma de tiempos 
    \item Se realizará un estudio del algoritmo secuencial con herramientas de profiling para detectar en que zonas del código se consume más tiempo, con el fin de optimizarlas.
    \item Partiendo del código secuencial y del estudio realizado con el profiler, se añadirán las directivas de OpenACC necesarias para tratar de optimizar el algortimo mediante el uso de la CPU multinúcleo.
    \item Se realizará una nueva etapa de experimentación y toma de tiempos
    \item Partiendo del código secuencial y del estudio realizado con el profiler, se utilizará la plataforma de programación CUDA, para la implementación de una versión para la GPU.
    \item Se realizará una última etapa de experimentación y toma de tiempos 
    \item Se realizará un análisis y discusión de los resultados obtenidos
    \item Se redactará una memoria (este documento) documentando todo el proceso seguido, que además, recogerá los resultados obtenidos y las conclusiones a las que se han llegado.
\end{itemize}
 
